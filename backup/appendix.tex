\section{ハミルトニアンの基底変換}
ここでは、Hamiltonianの変換を行う。
\begin{equation}
    \hat{\mathcal{H}}=\hbar\left(\hat{a}^{\dagger }\ \hat{b}^{\dagger }\right)\left(\begin{array}{cc}
    \tilde{\omega}_{a} & g(\Phi ) \\
    g(\Phi ) & \hat{\omega}_{b}
    \end{array}\right)\left(\begin{array}{l}
    \hat{a} \\
    \hat{b}
    \end{array}\right)
\end{equation}
\begin{equation}
    = \hbar \hat{\omega}_{a} \hat{a}^{\dagger} \hat{a}+\hbar \hat{\omega}_{b} \hat{b}^{\dagger} \hat{b}+\hbar g(\Phi)\left(\hat{a}^{\dagger}\hat{b}+\hat{a} \vec{b}^{+}\right)
\end{equation}
\begin{equation}
    \hat{c}_{\pm}=\frac{\hat{a} \pm \hat{b}}{\sqrt{2}} \quad \hat{c}_{+}^{\dagger}=\frac{\hat{a}^{\dagger} \pm \hat{b}^{\dagger}}{\sqrt{2}}
\end{equation}
\begin{equation}
    \hat{a}^{\dagger}=\frac{\hat{c}_{+}^{\dagger}+\hat{c}_{-}^{\dagger}}{\sqrt{2}} \quad \hat{b}=\frac{\hat{c}_{+}^{\dagger}-\hat{c}_{-}^{\dagger}}{\sqrt{2}}
\end{equation}
\begin{equation}
    \hat{a}=\frac{\hat{c}_{+}+\hat{c}}{\sqrt{2}} \quad \hat{b}=\frac{\hat{c}_{+}-\hat{c}_{-}}{\sqrt{2}}
\end{equation}
\begin{equation}
    \hat{a}^{\dagger} \hat{a}=\frac{1}{2}\left(\hat{c}_{+}^{\dagger}+\hat{c}_{-}^{+}\right)\left(\hat{c}_{t}+\hat{c}_{-}\right) \quad \hat{a}^{\dagger} \hat{b}=\frac{1}{2}\left(\hat{c}_{t}^{+}+\hat{c}_{-}^{+}\right)\left(\hat{c}_{+}-\hat{c}_{-}\right)
\end{equation}
\begin{equation}
    \hat{b} \hat{b}=\frac{1}{2}\left(\hat{c}_{t}+\hat{c}_{-}^{+}\right)\left(\hat{c}_{+}-\hat{c}_{-}\right) \quad \hat{a} \hat{b}^{\dagger}-\frac{1}{2}\left(\hat{c}_{t}+\hat{c}_{-}\right)\left(\hat{c}_{t}^{+}-\hat{c}_{-}^{+}\right)
\end{equation}
\begin{equation}
    \hat{a}^{\dagger} \hat{a}=\frac{1}{2}\left[\hat{c}_{t}^{+} \hat{c}_{+}+\hat{c}_{t}^{+} \hat{c}+\hat{c}_{-}^{+} \hat{c}_{t}+\hat{c}_{-}^{+} \hat{c}_{-}\right]
\end{equation}
\begin{equation}
    \hat{b}^{\dagger} \hat{b}=\frac{1}{2}\left[\hat{c}_{t}^{*} \hat{c}_{t}-\hat{c}_{t}^{n} \hat{c}-\hat{c}_{-}^{+} \hat{c}_{t}+\hat{c}_{-}^{+} \hat{c}-\right]
\end{equation}
\begin{equation}
    \hat{a}^{\dagger} \hat{b}=\frac{1}{2}\left[\dot{c}+\hat{c}_{t}-\hat{c}_{t} \hat{c}_{-}+\hat{c}_{-}^{+} \hat{c}_{t}-\hat{c}_{-}^{\prime} \hat{c}_{-}\right]
\end{equation}
\begin{equation}
    \hat{a} \hat{b}^{\dagger}=\frac{1}{2}\left[\hat{c}_{+} \hat{c}_{+}^{4}-\hat{c}_{+} \hat{c} \pm+\hat{c}-\hat{c}_{+}^{2}-\hat{c}-\hat{c}_{-}\right]
\end{equation}
\begin{equation}
    \hat{H}=\frac{\hbar}{2} \hat{w}_{a}\left[\hat{c}_{t}^{+} \hat{c}_{+}+\hat{c}_{t}^{+} \hat{c}+\hat{c}_{-}^{+} \hat{c}_{t}+\hat{c}_{-}^{+} \hat{c}_{-}\right]
\end{equation}
\begin{equation}
    +\frac{\hbar}{2} \hat{\omega}_{b}\left[\hat{c}_{+}^{+} \hat{c}_{+}-\hat{c}_{+}^{+} \hat{c}_{-}-\hat{c}_{-}^{+} \hat{c}_{+}+\hat{c}_{-}^{+} \hat{c}_{-}\right]
\end{equation}
\begin{equation}
    +\frac{\hbar}{2} g\left[2 \hat{c}+\hat{c}_{+}-2 \hat{c}_{-} \hat{c}\right]
\end{equation}
\begin{equation}
    \hat{H}=\frac{\hbar}{2}\left(\hat{\omega}_{a}+\hat{\omega}_{b}+2 g(\Phi)\right) \hat{c}_{t}^{+} \hat{c}_{+}+\frac{\hbar}{2}\left(\hat{w} a+\hat{w}_{b}-2 g(\Phi)\right) \hat{c}_{-}^{+} \hat{c}_{-}
\end{equation}
\begin{equation}
    +\frac{\hbar}{2} A\left(\hat{w}_{a}-\hat{w}_{b}\right)\left(\hat{c}_{t}+\hat{c}_{-}+\hat{c}_{+} \hat{c}_{-}\right)
\end{equation}
\begin{equation}
    \omega_{a}+\bar{c}_{b}+2 g(\Phi)=\Omega+
\end{equation}
\begin{equation}
    \omega_{a}+\bar{c}_{b}-2 g(\Phi)=\Omega-
\end{equation}
\begin{equation}
    \hat{\omega}_{a}-\hat{w}_{b}=\Delta
\end{equation}
\begin{equation}
    \hat{H}=\frac{\hbar}{2} \Omega+\hat{c}+\hat{c}+\frac{\hbar}{2} \Omega-\hat{c} \pm \hat{c}+\frac{\hbar}{2} \Delta\left(\hat{c}+\hat{c}+\hat{c}_{+} \hat{c}_{-}^{+}\right)
\end{equation}
\begin{equation}
    -\frac{\hbar}{2}\left(\begin{array}{cc}
    \hat{c}_{+} & \hat{c} \pm
    \end{array}\right)\left(\begin{array}{cc}
    \Omega_{1} & \Delta \\
    2 & \Omega_{2}
    \end{array}\right)\left(\begin{array}{l}
    \hat{c}_{+} \\
    \hat{c}_{-}
    \end{array}\right)
\end{equation}
\section{ミアンダインダクタンス}
\section{rf-SQUIDの相互インダクタンス}
dc-SQUIDのインダクタンスは
\begin{equation}
    L_{s}(\Phi)=\frac{\Phi_0}{4\pi I_{c}|{\cos({\phi_{-}^{min}(\Phi_{ext})}})|}
\end{equation}
と記述することができる。
\begin{equation}
    \Phi=\Phi_{ext}+L_{loop}
\end{equation}
\begin{equation}
    \beta_{dc}=\frac{2\pi L_{loop} I_{c}}{\Phi_{0}}
\end{equation}
とすると。
\section{マスター方程式}
\section{2点相関関数}
