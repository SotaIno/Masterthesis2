\section{周波数領域測定}
    \subsection{One-tone spectroscopy}
最初に、One-Tone spectroscopyという実験によって共振器の周波数を測定した。
共振器の入力ポートに対してVNAから周波数を変えながらマイクロ波信号を送信し、共振器読み出しポートから帰ってくる信号の透過係数$S21$を観察する。
\subsubsection{Power sweep}
Power sweepのOne-Tone spectroscopyでは、共振器に入力するマイクロ波のパワーを調整することによって
共振器の周波数特性が変わることを利用する。

\subsection{Two-tone spectroscopy}
(Continuous) Two-tone spectroscopyでは、One-tone spectroscopy から得た共振器1のVNAからの信号を
    \subsubsection{Drive power sweep}
    \begin{figure}[H]
        \begin{center}
            \includegraphics[width=11cm]{Q1side_twotone.png}
            \caption{量子ビット1}
        \end{center}
    \end{figure}
    \subsubsection{Current sweep}


    \begin{figure}[H]
        \begin{center}
            \includegraphics[width=12cm]{Coupling_between_Qubits.png}
            \caption{11準位,02準位と思われる準位間の準位反発}
        \end{center}
    \end{figure}