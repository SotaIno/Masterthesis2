\section*{序章}

現在のコンピュータを上回る計算性能を秘めているとして近年注目を集めている量子コンピュータは、大規模化・高精度化が進んでいる。非常に限定された分野においてのみではあるが、ここ数年ではスーパーコンピュータが100年以上掛かって解ける問題を数分の間に解けるような実機も登場している。
一昨年、53個の超伝導量子ビットを実装した量子計算機によるQuantum supremacy(量子超越性)がGoogleの研究チームにより報告された。
\cite{arute2019quantum} またこちらは実装手段に光を用いているが、昨年12月に中国の研究グループによりガウシアンボソンサンプリングという問題に関しての量子超越性が報告された。\cite{zhong2020quantum}\\
量子コンピュータ内で行われる計算の方式には大きく分けて2通りが存在する。
その一つのゲート型方式は、一個または複数個の量子ビットに対し「量子ゲート」と呼ばれる演算を逐次実行し計算を行う。
もう一つはアニーリング方式とよばれる、解く問題をイジングモデルに帰着させそのエネルギーが最小となるパラメータを求める方式である。\\
前者のゲート型は、論理ゲートを用いる現在のNeumann型コンピュータに近い回路モデルで計算を行う。そのためアニーリング型に比べ汎用性の高い方式とみられている一方で、ゲート型ならではの課題もある。
量子ゲートは論理ゲートと異なり、目的とする演算を常に100\%再現することはできず
本紙では、量子ビット数の拡張を行う上で鍵となる2量子ビット間CZゲートについて行ったシミュレーション、及び超伝導体を用いて作成したサンプルに関する測定を行って得られた結果について報告する。
断熱的・非断熱的なCZゲートについて述べる。